%!TEX root = /Users/ede/Documents/Master/19_AS/Ausarbeitung/as-ausarbeitung.tex
%
%
% Titelseite der Arbeit
%
%

\begin{titlepage}
\center
\vspace*{4em}
\large \textbf{\textsc{Advanced Seminar}}\\ 
\vspace{1em}
%\huge\textbf{{\sffamily Extraction von Verkehrsinformationen aus GPS-Logs}}
\huge\textbf{{\sffamily Tag-Rankingverfahren für Bilder}}
\vspace{3em}

\large
\textsc{Dozent}\\
 Prof. Dr. Kristian Fischer \\
\vspace{1em}
 Fachhochschule Köln, Campus Gummersbach\\
\vspace{8em}







% \vspace{8em}
%  \large Tim Haussmann\\
%  \large mim103@gm.fh-koeln.de\\
\vspace{17em}
Köln, \today
\end{titlepage}

\begin{abstract}
Die semantische Beschreibung von Daten im Internet gewinnt immer mehr Bedeutung. Tagging als einfache Variante der Ontologie-freien Beschreibung erfreut sich in vielen Bereichen großer Beliebtheit. Es erlaubt effektiveres Suchen und Einordnen im Vergleich zu bisherigen Suchansätzen. Vor allem bei sozialen Netzwerken wie Flickr entstehen viele Annotationen. Dabei müssen die Tags nach ihrer Relevanz zum Inhalt des Bildes priorisiert werden. Ranking Verfahren beschreiben Möglichkeiten, die ungeordneten Tags mit Relevanzinformationen anzureichern und damit zu bewerten. 
\end{abstract}