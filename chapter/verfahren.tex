%!TEX root = /Users/ede/Documents/Master/19_AS/Ausarbeitung/as-ausarbeitung.tex
\section{Tag-Ranking Verfahren} % (fold)
\label{sec:tag_ranking_verfahren}
Hier erfolgt die Darstellung der Tag-Ranking Verfahren im genauen.

% 
% 
% \begin{itemize}
%   \item   Visual diversification of image search results \cite{diversification}
%   \item   Tag ranking \cite{ranking}
%   \item   Learning to tag \cite{learningToTag}
%   \item   Learning tag relevance by neighbor voting for social image retrieval \cite{learningtagrelevance}
%   \item   Improving recommendation lists through topic diversification \cite{improvingRecommendations}
%   \item   Why we tag: motivations for annotation in mobile and online media \cite{whyWeTag}
%   \item   Flickr tag recommendation based on collective knowledge \cite{collectiveKnowledge}
% \end{itemize}

\subsection{Ranking basierend auf kollektivem Wissen nach Sigurbjörnsson und van Zwol} % (fold)
\label{sub:ranking_basierend_auf_kollektivem_wissen_nach_zwol_et_al_}

\begin{itemize}
  \item Tag co-occurrence is the key to our tag recommendation approach, and only works reliable when a large quantity of supporting data is available.
  \item We define the co-occurrence between two tags to be the number of photos [in our collection] where both tags are used in the same annotation.
  \item Symmetric measures vs. Asymmetric measures
  \item Zweiter Schritt: Tag Aggregation and Promotion
  \begin{itemize}
    \item Aggregation durch \emph{Vote} und \emph{Sum} Verfahren
    \item Priorisierung der Tags durch \emph{Stability-promotion} und \emph{Descriptiveness-promotion}
  \end{itemize}
\end{itemize}

\begin{figure}[htbp]
  \centering
    \includegraphics[height=3in]{images/collective_knowledge_system_overview.png}
  \caption{System overview of the tag recommendation process aus \cite{collectiveKnowledge}}
  \label{fig:images_collective_knowledge_system_overview}
\end{figure}


Ausführliche Beschreibung des Verfahrens folgt.

% subsection ranking_basierend_auf_kollektivem_wissen_nach_zwol_et_al_ (end)

\subsection{Verbesserung der Relevanz von Tags durch einen Random Walk nach Liu u. a.} % (fold)
\label{sub:verbesserung_der_relevanz_durch_einen_random_walk}

\begin{itemize}
  \item Wahrscheinlichkeitsorientierte Schätzung der Relevanz von Tags.
  \item Random-Walk basierte Verfeinerung des Rankings
    
    \begin{itemize}
      \item Aufbau eines Beziehungs-Graphen der Tags.
      \item Random Walk über den Graphen
    \end{itemize}
\end{itemize}

\begin{figure}[htbp]
  \centering
    \includegraphics[height=2in]{images/tag_ranking_verfahren.png}
  \caption{The illustrative scheme of the tag ranking approach. A probabilistic method is first adopted to estimate tag relevance score. Then a random walk-based refinement is performed along the tag graph to further boost tag ranking performance aus \cite{ranking}}
  \label{fig:images_tag_ranking_verfahren}
\end{figure}


Ausführliche Beschreibung des Verfahrens folgt.

% subsection verbesserung_der_relevanz_durch_einen_random_walk (end)
% section tag_ranking_verfahren (end)
