%!TEX root = /Users/ede/Documents/Master/19_AS/Ausarbeitung/as-ausarbeitung.tex
\section{Fazit}
In dieser Arbeit wurde vorgestellt, warum die manuelle Annotation von Multimedia, das Tagging, zur Zeit noch die sinnvollere Variante der Metadatenerstellung ist und welche Potentiale daraus für die Suche nach Multimedia entstehen. Die Analyse von Tags der Folksonomy von Flickr offenbart mehrere Probleme die durch das Tagging entstehen, welche jedoch durch die Tag Ranking Verfahren reduziert werden sollen. Gleichzeitig verspricht die Anwendung dieser Verfahren eine bessere Sortierung der Ergebnisse einer Suche nach ihrer Relevanz und einen einfacheren Annotationsprozess für den Benutzer. %Die momentan in Flickr vorhandenen Tags enthalten keine Relevanzwerte

Es werden also vielfältige Anforderungen an ein Tag Ranking Verfahren gestellt, so dass von den Autoren zunächst eine Analyse der bereits vorhanden Tags durchgeführt wurde. Die Analyse beantwortet Fragen über die Häufigkeitsverteilung von Tags, die Beziehungen der Tags über die Photos, welche Art von Inhalten die Benutzer annotieren und weiter statistische Daten. Basierend darauf erfolgt eine Klassifikation der Photos und Tags, welche in der späteren Evaluation Verwendung findet.

Der Ansatz von Sigurbjörnsson und van Zwol fokussiert auf den Wert der co-occurrence von zwei Tags. Basierend darauf wurden zwei Aggregationsstrategien vorgestellt, um die Kandidatensequenzen für benutzerdefinierte Tags zu ermitteln und zu einer, nach der Relevanz sortierten Sequenz, zusammen zu führen. Daraufhin erfolgt die Einbeziehung der Analyse, wobei extrem hoch- und niederfrequente Tags abgewertet werden. Das Verfahren eignet sich damit zum Vorschlagen von Tags bei der Annotation von Photos durch den Benutzer.

Die Autoren von \cite{ranking} gehen einen anderen Weg und schätzen zunächst mit Hilfe einer stochastischen Methodik einen initialen Ranking Wert für jeden Tag eines Photos. Daraufhin wird ein Tag Graph konstruiert, wobei die Ähnlichkeit von Photos mit einbezogen wird und ein Random Walk ausgeführt wird. Die Methode des Random Walk ist in dem Gebiet des Information Retrieval etabliert und resultiert in guten Ergebnissen bei der Ermittlung von Relevanzwerten durch die Beziehung der einzelnen Tags untereinander.

Beide Arbeiten werden durch eine ausführliche Evaluation der Ranking Verfahren abgeschlossen. Hierin weisen Sie nach, dass die Verwendung der vorgestellten Ranking Verfahren deutliche Verbesserungen in den Anwendungsgebieten von Tags, wie dem Vorschlagen von Tags bei der Annotation oder der tagbasierten Suche nach Multimedia Objekten, zur Folge hat.